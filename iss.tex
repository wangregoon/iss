\documentclass{paper}

\usepackage[margin=1in]{geometry} 
\usepackage{amsmath,amsthm,amssymb}
\usepackage{textcomp}
\usepackage[colorlinks,linkcolor=red,anchorcolor=green,citecolor=blue]{hyperref}
\usepackage{picture}

% topic specific 
\newcommand{\diff}{\text{d}}
\newcommand{\R}[1]{\mathbb{R}^{#1}}
\newcommand{\D}{\mathcal{D}}
\newcommand{\A}{\mathcal{A}}
\newcommand{\K}{\mathcal{K}}
\newcommand{\KL}{\mathcal{KL}}
\newcommand{\KI}{\mathcal{K}_\infty}
\newcommand{\MD}{\mathcal{M}_\mathcal{D}}
\newcommand{\tb}[1]{\textbf{#1}}
\newcommand{\ti}[1]{\textit{#1}}
\newcommand{\mca}[1]{\mathcal{#1}}
\newcommand{\map}[3]{{#1}:{#2}\rightarrow{#3}}

% common 
\newtheorem{problem}{Problem}
\newtheorem{thm}{Theorem}[section]
\newtheorem{prop}{Proposition}[section]
\newtheorem{lem}{Lemma}[section]
\newtheorem{cor}{Coro.}[section]
%\newtheorem{conj}[thm]{Conjecture}
\newtheorem{exer}{Exer}[section]

\theoremstyle{definition}
\newtheorem{defi}{Defination}[section]
%\newtheorem{example}{Example}[section] 

\theoremstyle{remark}
\newtheorem*{note}{Note}
\newtheorem{assume}{\tb{Assumption}}
\newtheorem*{remark}{\tb{Remark}}
\newtheorem*{claim}{\tb{Claim}}
\newtheorem*{examp}{\tb{Example}}
\newenvironment{solution}
               {\let\oldqedsymbol=\qedsymbol
                \renewcommand{\qedsymbol}{$\blacktriangleleft$}
                \begin{proof}[\bfseries\upshape Solution]}
               {\end{proof}
                \renewcommand{\qedsymbol}{\oldqedsymbol}}
 
\begin{document}
 
\title{Literature Review on Input State Stability}
\author{Regoon Wang, ChemE@UNSW} 

\maketitle
\begin{abstract}
The Input-to-State Stability (ISS) property provides a natural framework in which to formulate notions
of stability with respect to input perturbations. This works reviews ISS theory \cite{iss:1,iss:2,iss:3,iss:4} 
and its application in MPC \cite{iss:5,iss:6}. 
\end{abstract} 

\tableofcontents
\section{Smooth Converse Lyapunove Theorem}
Smooth converse Lyapunove theorem is the cornerstone of ISS. Most of the proofs of ISS theorems and lemmas depend 
on this theorem or the technique to prove this theorem. Consider the following system:
\begin{equation}\label{eq:1}
\dot{x}(t)=f(x(t),d(t))
\end{equation}
where $x(t)\in\R{n}$ and $d(d)\in\D$, and where $\D$ is a compact subset of $\R{m}$. The map $\map{f}{\R{n}\times\D}
{\R{n}}$ is assumed to satisfy the following two properties:
\begin{itemize}
\item $f$ is continuous.
\item $f$ is locally Lipschitz on $x$ uniformly on d, that is, for each compact subset K there is some constant c
s.t. $|f(x,d)-f(z,d)|\leqslant c|x-z|$.
\end{itemize}
Let $\MD$ be the set of all measurable functions and $d(t)\in\MD$. Let $x(t)=x(t,x_0,d)$ be the solution on some
maximal interval $\left(T_{x_0,d}^-,T_{x_0,d}^+\right)$. The system is said to be forward complete if $T_{x_0,d}^+=+\infty$,
backword complete if $T_{x_0,d}^-=-\infty$ and complete if it is both forward and backward complete. A closed set 
$\A $ is an invariant set for Eq.\ref{eq:1} if
\begin{equation*}
\forall x_0\in\A,\forall d\in\MD,T_{x_0,d}^+=+\infty\quad \text{and}\quad x(t,x_0,d)\in\A,\quad\forall t\geqslant 0
\end{equation*}
Define point-to-set distance $|\xi|_\A:=\underset{\eta\in\A}{\inf}d(\xi,\eta)$.
\begin{defi}
Eq.\ref{eq:1} is uniformly globally asymptotically stable (UGAS) with respect to the closed invariant set $\A$ if 
it is forward complete and the following two properties hold:
\begin{enumerate}
\item[(i)] \tb{Uniform stability}: $\forall\epsilon\geqslant 0,\exists \delta(\epsilon)\in\KI,\quad |x_0|_\A\leqslant
\delta(\epsilon)\Rightarrow |x(t)|_\A<\epsilon$
\item[(ii)] \tb{Uniform Attraction}: $\forall d\in\MD, \forall r,\epsilon>0,\exists T>0,\quad |x_0|_\A<r,t\geqslant 
T\Rightarrow |x(t)|_\A<\epsilon$
\end{enumerate}
\end{defi}
\begin{defi}
A Lyapunov function for Eq.\ref{eq:1} with respect to a nonempty, closed, invariant set $\A$ is a function $\map{V}
{\R{n}}{\mathcal{R}}$ s.t. V is smooth on $\R{n}\backslash\A$ and satisfies
\begin{enumerate}
\item[(i)] $\exists \alpha_1,\alpha_2\in\KI,\alpha_1(|\xi|_\A)\leqslant V(\xi)\leqslant\alpha_2(|\xi|_\A)$
\item[(ii)] $\exists \alpha_3\in\K, L_{f_d}V(\xi)\leqslant-\alpha_3(|\xi|_\A)$
\end{enumerate}
\end{defi}
\begin{thm}
Assume that the Eq.\ref{eq:1} is complete. Let $\A$ be a nonempty, closed, invariant subset for this system. This system 
is UGAS with respect to $\A$ iff there exist a smooth Lyapunov function V with respect to $\A$.
\end{thm}
\begin{thm}
Let $\A$ be a nonempty, compact, invariant subset. This system is UGAS with respect to $\A$ iff there exist a smooth 
Lyapunov function V with respect to $\A$.
\end{thm}
\section{ISS Theory}
\subsection{Four Definitions of ISS}
Four natural definitions of ISS are proposed and proved to be  equivalent.
\begin{enumerate}
\item[(i)] \tb{From GAS to ISS}. The GAS requires the system be complete and the following two properties hold: 1.(stability),
the map $x_0\longmapsto x(\cdot)$ is continuous at 0; 2.(attractivity), $\lim\limits_{t\rightarrow +\infty}|x(t,x_0)|=0$. By
analogy, the ISS requires the system be complete and the following two properties hold: 1. the map $(x_0,u)\longmapsto x(\cdot)$
is continuous at $(0,0)$; 2. there exists a nonlinear asymptotic gain $\gamma\in\K$ s.t. $\overline{\lim}_{t\rightarrow
+\infty}|x(t,x_0,u)|\leqslant\gamma(\|u\|_\infty)$.
\item[(ii)] \tb{From Lyapunov to Dissipation}. A Lyapunov function for the zero-input system is a storage function for which there
exists $\alpha\in\KI$ s.t. $\nabla V(x)\cdot f_0(x)\leqslant-\alpha(|x|)$. By analogy, when nonzero inputs must be taken into
account, it is sensible to define an ISS-Lyapunove function as a storage function for which there exist $\alpha,\theta\in\KI$ 
s.t. $\nabla V(x)\cdot f(x,u)\leqslant\theta(|u|)-\alpha(|x|)$. Let $w(u,x)=\theta(|u|)-\alpha(|x|)$, it turns out to be a
dissipation inequality.
\item[(iii)] \tb{Gain Margins}. A stability margin for system is defined as any $\rho\in\KI$ with the following property: for 
each admissible feedback law $k(t,x)\leqslant \rho(|x|)$, the closed-loop system $\dot{x}=f(x,k(t,x))$ is GAS, uniformly on k.
\item[(iv)] \tb{Estimates}. There exist $\beta\in\KL$ and $\gamma\in\K$ s.t. $|x(t,x_0,u)|\leqslant\beta(|x_0|,t)+\gamma(\|u\|_\infty)$.
\end{enumerate}
\subsection{Checking the ISS}
\begin{thm}
Consider a storage function with the property that there exist $\alpha,\chi\in\K$ s.t. the implication 
\begin{equation}
|x|\geqslant\chi(|u|)\Rightarrow\nabla V(x)\cdot f(x,u)\leqslant-\alpha(|x|)
\end{equation}
holds for all $x,u$, then the system is ISS.
\end{thm}

\section{Application in MPC}

\bibliographystyle{unsrt}
\bibliography{iss} 

\end{document}